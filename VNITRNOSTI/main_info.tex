\cleardoublepage
\thispagestyle{empty}
%\thisfancypage{\setlength{\fboxsep}{5pt}\doublebox}{}
%\setlength\intextsep{0mm}
\noindent
\includegraphics[width=0.45\textwidth]{IMG/TOP/logoVSCHT_zakl_CB.png} \\
\vspace{10mm}
\\
{\Large \textbf{Fakulta  chemicko-inženýrská}
\\ [5mm]
Ústav počítačové a řídicí techniky}

\vspace{30mm}


\begin{spacing}{0.9}
\Huge\noindent POUŽITÍ ADAPTIVNÍCH SYSTÉMŮ PŘI ANALÝZE DAT\\ 
\end{spacing}
\vspace{20mm}

\noindent
{\Large \textbf{TEZE DISERTAČNÍ PRÁCE}} 

\vspace{10mm}

\begin{table}[!h]
\begin{tabular}{  l l |l  l }
\hspace{-0.5em}AUTOR & \hspace{0mm} & & {\Large \textbf{Ing. Jan Vrba}} \\ [5mm]
\hspace{-0.5em}ŠKOLITEL &  &  & \textbf{\large doc. Ing. Jan Mareš, Ph.D.}\\ [5mm]
\hspace{-0.5em}ŠKOLITEL  SPECIALISTA             &     &   & {\textbf{\large doc. Ing. Pavel Hrnčiřík, Ph.D. }} \\ [5mm]
\hspace{-0.5em}STUDIJNÍ PROGRAM &  &  & {\large Chemické a procesní inženýrství (čtyřleté)} \\ [5mm]
\hspace{-0.5em}STUDIJNÍ OBOR    & &   & {\large Technická kybernetika}\\ [5mm]
\hspace{-0.5em}ROK          &       &   & \textbf{2020} 
\end{tabular}


\end{table}

%%====================================================Declaration=====================
%\newpage
%\thispagestyle{empty}
%\vphantom{a}
%\vspace{13cm}
%\begin{figure}[H]
%\pdfimageresolution=133 \includegraphics{IMG/TOP/QR.png}
%\end{figure}




%-------------------------------Abstract CZ ---------------------------------------------
\clearpage
\thispagestyle{empty}
%-------------------------------Abstract CZ ---------------------------------------------

\noindent {\bf \large Souhrn} \\ [5mm]
Dizertační práce se zabývá použitím adaptivních systémů v oblasti detekce novosti. Tento přístup v oblasti detekce novosti v datech se stal v posledních letech slibným směrem výzkumu. V rámci této práce je navržen nový algoritmus pojmenovaný jako Extreme Seeking Entropy. Tento algoritmus je založen na vyhodnocování přírůstku adaptivních parametrů systémů pomocí zobecněného Paretova rozdělení. Navržený algoritmus byl otestován na celé řadě typů syntetických dat, která reprezentují různé druhy novosti. Pro detekci změny trendu a skokové změny generátoru signálu pak bylo provedeno i vyhodnocení úspěšnosti detekce. Dále byla provedena experimentální studie zabývající se časovou náročností výpočtu algoritmu, vyhodnocena ROC (Receiver Operating Characteristic) křivka pro detekci změny trendu, a provedena studie detekce epilepsie v záznamu EEG myši.\\ [5mm]
\noindent {\bf \large Klíčová slova}  \\ [5mm] \noindent {\it adaptivní systémy, detekce novosti, časové řady, extreme seeking entropy} \\ [5mm]

\noindent {\bf \large Summary} \\ [5mm] 
The dissertation deals with the use of adaptive systems in the field of novelty detection. The use of adaptive systems for detecting novelty in data has become a promising direction of research in recent years. In this work, a new algorithm was designed using adaptive systems, called Extreme Seeking Entropy. This algorithm is based on evaluating the increment of adaptive parameters of systems using a generalized Pareto distribution. The proposed algorithm has been tested on a number of types of synthetic data that represent different types of novelty. To detect a change in the trend and a step-change in the signal generator, an evaluation of the detection success was performed. Furthermore, an experimental study was performed dealing with the time consumption of the algorithm, the ROC curve was evaluated to detect a change in the trend, and a study was performed to detect epilepsy in the EEG mouse record.
\\ [5mm]



\noindent {\bf \large Keywords}  \\ [5mm]
{\it adaptive systems, novelty detection, time series, extreme seeking entropy}

\cleardoublepage
\thispagestyle{empty}








%%v~v~v~v~v~v~v~v~v~v~v~v~v~v~v~v~v~v~v~v~v~v~v~v~v~v~v~v~v
%\thispagestyle{empty}
%%v~v~v~v~v~v~v~v~v~v~v~v~v~v~v~v~v~v~v~v~v~v~v~v~v~v~v~v~v

\tableofcontents

%\addtocontents{toc}{~\hfill\textbf{Page}\par}
%\addcontentsline{toc}{chapter}{Contents}
%%v~v~v~v~v~v~v~v~v~v~v~v~v~v~v~v~v~v~v~v~v~v~v~v~v~v~v~v~v

%\newpage
%\setcounter{page}{1}
