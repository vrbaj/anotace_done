\documentclass[11pt,twoside,openright]{report}
\usepackage[a4paper,width=155mm, inner=25mm, top=30mm,bottom=40mm, marginparsep=0mm]{geometry}
\usepackage[utf8]{inputenc} 
\usepackage[T1]{fontenc} 
\usepackage{longtable}
\usepackage[czech]{babel} 
\usepackage{graphicx}
\usepackage{emptypage} % maže číslování na prázdných stranách
\usepackage[numbers,sort&compress]{natbib}
\usepackage{formatting,amsfonts,amsmath,amssymb}
\usepackage{booktabs,mathtools}
\usepackage{lipsum,url}
\usepackage[unicode]{hyperref}
\usepackage{textcomp}
\usepackage{multirow}
\usepackage{setspace}
\usepackage{array}
\usepackage{float}
\usepackage{mwe}
\usepackage{appendix}
\usepackage{pdfpages}
\usepackage{etoolbox}
\usepackage[nottoc]{tocbibind}
\usepackage{tikz}
%\usepackage{enumerate}
\usepackage{enumitem}
\usepackage{algorithmic}
\usepackage{algorithm}
\usepackage{csquotes}
\usetikzlibrary{shapes, arrows.meta}
\usetikzlibrary{backgrounds,calc}
\DeclarePairedDelimiter{\ceil}{\lceil}{\rceil}
\BeforeBeginEnvironment{appendices}{\gdef\thechapter{\ }}
\AtBeginEnvironment{appendices}{\renewcommand\thesection{\Alph{section}}}

% \cmidrule have problem with Czech
\usepackage{regexpatch}
\makeatletter
% Change the `-` delimiter to an active character
\xpatchparametertext\@@@cmidrule{-}{\cA-}{}{}
\xpatchparametertext\@cline{-}{\cA-}{}{}
\makeatother

%\usepackage[nottoc,numbib]{tocbibind}
\usepackage{fancyhdr}
\usepackage{booktabs}
\usepackage{subfig}
\usepackage{fancybox}

\DeclarePairedDelimiter\abs{\lvert}{\rvert}%
\DeclarePairedDelimiter\norm{\lVert}{\rVert}%
\makeatletter
\renewcommand{\ALG@name}{Algoritmus}
\makeatother
\pagestyle{fancy}
%\renewcommand{\chaptermark}[1]{%
%  \markboth{\MakeUppercase{\thechapter\ \quad  #1}}{}% 
%}
\renewcommand{\chaptermark}[1]{%
  \markboth{\MakeUppercase{#1}}{}% 
}
\fancyfoot{}{}{}
\fancyhead[LO,LE]{}
\fancyhead[RO]{\leftmark}
\fancyhead[RE]{}
\fancyfoot[RO]{\vspace*{0.8\baselineskip}\thepage}
\fancyfoot[LE]{\vspace*{0.8\baselineskip}\thepage}
\renewcommand{\headrulewidth}{0pt}
\renewcommand{\footrulewidth}{0pt}

\fancypagestyle{plain}{%
\fancyhead[RO]{}
\fancyhead[LO,LE]{}
\fancyfoot[RO]{\vspace*{0.8\baselineskip}\thepage}
\fancyfoot[LE]{\vspace*{0.8\baselineskip}\thepage}
\renewcommand{\headrulewidth}{0pt}
\renewcommand{\footrulewidth}{0pt}
}
\makeatletter
\renewcommand\@dotsep{300}   % default value 4.5
\makeatother

\linespread{1.25}
\let\cleardoublepage=\clearpage
\begin{document}
\pagenumbering{gobble}

\cleardoublepage
\thispagestyle{empty}
%\thisfancypage{\setlength{\fboxsep}{5pt}\doublebox}{}
%\setlength\intextsep{0mm}
\noindent
\includegraphics[width=0.45\textwidth]{IMG/TOP/logoVSCHT_zakl_CB.png} \\
\vspace{10mm}
\\
{\Large \textbf{Fakulta  chemicko-inženýrská}
\\ [5mm]
Ústav počítačové a řídicí techniky}

\vspace{30mm}


\begin{spacing}{0.9}
\Huge\noindent POUŽITÍ ADAPTIVNÍCH SYSTÉMŮ PŘI ANALÝZE DAT\\ 
\end{spacing}
\vspace{20mm}

\noindent
{\Large \textbf{TEZE DISERTAČNÍ PRÁCE}} 

\vspace{10mm}

\begin{table}[!h]
\begin{tabular}{  l l |l  l }
\hspace{-0.5em}AUTOR & \hspace{0mm} & & {\Large \textbf{Ing. Jan Vrba}} \\ [5mm]
\hspace{-0.5em}ŠKOLITEL &  &  & \textbf{\large doc. Ing. Jan Mareš, Ph.D.}\\ [5mm]
\hspace{-0.5em}ŠKOLITEL  SPECIALISTA             &     &   & {\textbf{\large doc. Ing. Pavel Hrnčiřík, Ph.D. }} \\ [5mm]
\hspace{-0.5em}STUDIJNÍ PROGRAM &  &  & {\large Chemické a procesní inženýrství (čtyřleté)} \\ [5mm]
\hspace{-0.5em}STUDIJNÍ OBOR    & &   & {\large Technická kybernetika}\\ [5mm]
\hspace{-0.5em}ROK          &       &   & \textbf{2020} 
\end{tabular}


\end{table}

%%====================================================Declaration=====================
%\newpage
%\thispagestyle{empty}
%\vphantom{a}
%\vspace{13cm}
%\begin{figure}[H]
%\pdfimageresolution=133 \includegraphics{IMG/TOP/QR.png}
%\end{figure}




%-------------------------------Abstract CZ ---------------------------------------------
\clearpage
\thispagestyle{empty}
%-------------------------------Abstract CZ ---------------------------------------------

\noindent {\bf \large Souhrn} \\ [5mm]
Dizertační práce se zabývá použitím adaptivních systémů v oblasti detekce novosti. Tento přístup v oblasti detekce novosti v datech se stal v posledních letech slibným směrem výzkumu. V rámci této práce je navržen nový algoritmus pojmenovaný jako Extreme Seeking Entropy. Tento algoritmus je založen na vyhodnocování přírůstku adaptivních parametrů systémů pomocí zobecněného Paretova rozdělení. Navržený algoritmus byl otestován na celé řadě typů syntetických dat, která reprezentují různé druhy novosti. Pro detekci změny trendu a skokové změny generátoru signálu pak bylo provedeno i vyhodnocení úspěšnosti detekce. Dále byla provedena experimentální studie zabývající se časovou náročností výpočtu algoritmu, vyhodnocena ROC (Receiver Operating Characteristic) křivka pro detekci změny trendu, a provedena studie detekce epilepsie v záznamu EEG myši.\\ [5mm]
\noindent {\bf \large Klíčová slova}  \\ [5mm] \noindent {\it adaptivní systémy, detekce novosti, časové řady, extreme seeking entropy} \\ [5mm]

\noindent {\bf \large Summary} \\ [5mm] 
The dissertation deals with the use of adaptive systems in the field of novelty detection. The use of adaptive systems for detecting novelty in data has become a promising direction of research in recent years. In this work, a new algorithm was designed using adaptive systems, called Extreme Seeking Entropy. This algorithm is based on evaluating the increment of adaptive parameters of systems using a generalized Pareto distribution. The proposed algorithm has been tested on a number of types of synthetic data that represent different types of novelty. To detect a change in the trend and a step-change in the signal generator, an evaluation of the detection success was performed. Furthermore, an experimental study was performed dealing with the time consumption of the algorithm, the ROC curve was evaluated to detect a change in the trend, and a study was performed to detect epilepsy in the EEG mouse record.
\\ [5mm]



\noindent {\bf \large Keywords}  \\ [5mm]
{\it adaptive systems, novelty detection, time series, extreme seeking entropy}

\cleardoublepage
\thispagestyle{empty}








%%v~v~v~v~v~v~v~v~v~v~v~v~v~v~v~v~v~v~v~v~v~v~v~v~v~v~v~v~v
%\thispagestyle{empty}
%%v~v~v~v~v~v~v~v~v~v~v~v~v~v~v~v~v~v~v~v~v~v~v~v~v~v~v~v~v

\tableofcontents

%\addtocontents{toc}{~\hfill\textbf{Page}\par}
%\addcontentsline{toc}{chapter}{Contents}
%%v~v~v~v~v~v~v~v~v~v~v~v~v~v~v~v~v~v~v~v~v~v~v~v~v~v~v~v~v

%\newpage
%\setcounter{page}{1}

% TAK AT TO JDE OD RUKY

\input{VNITRNOSTI/uvod}

\input{VNITRNOSTI/ese}
\input{VNITRNOSTI/vysledky}
\input{VNITRNOSTI/zaver}

\renewcommand{\bibname}{Publikace autora}
\begin{thebibliography}{A}
\end{thebibliography}

\begin{enumerate}[label={[V\arabic*]}]
    \item \label{ese_mdpi} VRBA, Jan; MAREŠ, Jan. Introduction to Extreme Seeking Entropy. \textit{Entropy}, 2020, 22.1: 93.
  	\item \label{appel2}VRBA, Jan; MAREŠ, Jan. Computational Performance of the Parameters Estimation in Extreme Seeking Entropy Algorithm. In: \textit{2020 International Conference on Applied Electronics (AE)}. IEEE, 2020. p. 1-4.
	\item \label{appel3}VRBA, Jan; MAREŠ, Jan. ROC Analysis of Extreme Seeking Entropy for Trend Change Detection. In: \textit{2020 International Conference on Applied Electronics (AE)}. IEEE, 2020. p. 1-4.  
	\item \label{artep}VRBA, Jan. Využití fuzzy systémů a algoritmu learning entropy pro detekci změn stavů bioprocesu. In: \textit{Automatizacia a riadenie v teorii a praxi ARTEP 2017}. Technická univerzita Košice, 2017.

	\item \label{asr} VRBA, Jan. \textit{XLIII. Seminář ASŘ - Adaptivní metoda detekce} [přednáška]. Ostrava: VŠB TU Ostrava, 27.4.2018.
    \item \label{appel1}VRBA, Jan. Adaptive Novelty Detection with Generalized Extreme Value Distribution. In: \textit{2018 International Conference on Applied Electronics (AE)}. IEEE, 2018. p. 1-4.
    \item \label{ijcnn}BUKOVSKÝ, Ivo, et al. Study of learning entropy for onset detection of epileptic seizures in EEG time series. In: \textit{2016 International Joint Conference on Neural Networks (IJCNN)}. IEEE, 2016. p. 3302-3305.

    
    \item \label{cyril}OSWALD, Cyril, et al. Novelty Detection in System Monitoring and Control with HONU. In: \textit{Applied Artificial Higher Order Neural Networks for Control and Recognition}. IGI Global, 2016. p. 61-78.


	\item \label{roboti}VRBA, Jan, et al. An Automated Platform for Microrobot Manipulation. In: \textit{International Workshop on Soft Computing Models in Industrial and Environmental Applications}. Springer, Cham, 2020. p. 255-265.
	\item \label{bila1}BÍLA, Jiří; VRBA, Jan. The Detection and Interpretation of Emergent Situations in ECG Signals. In: \textit{International Conference on Soft Computing-MENDEL}. Springer, Cham, 2016. p. 264-275.

	\item \label{ijcnn3}MOJZES, Matej, et al. Feature selection via competitive levy flights. In: \textit{2016 International Joint Conference on Neural Networks (IJCNN)}. IEEE, 2016. p. 3731-3736.
	
	\item \label{bila2}BÍLA, Jiří; NOVÁK, Martin; VRBA, Jan. Detection of emergent situations in complex systems represented by algebras of transformations. In: \textit{MATEC Web of Conferences}. EDP Sciences, 2016. p. 02035.	
	\item \label{artep2}BUKOVSKÝ, Ivo, OSWALD, Cyril, VRBA, Jan. Případová studie použití entropie učení pro adaptivní detekci při řízení spalování tuhých paliv. In: \textit{Automatizácia a riadenie v teórii a praxi 2015}. Technická univerzita Košice, 2015.
\end{enumerate}

\renewcommand{\bibname}{Literatura relevantní k tezím}
\begin{thebibliography}{L}
\bibitem{expo}SHARMA, Anish and ANDREWS, Rebecca. Managing-Exponential-Data-Growth-and-Application-Modernization. \textit{IBM} [online]. 11 November 1999. [cit. 11.8.2020]. Dostupné z: https://www.ibm.com/cloud/blog/managing-exponential-data-growth-and-application-modernization 
\bibitem{ivoLE1}BUKOVSKY, Ivo. Learning entropy: Multiscale measure for incremental learning. \textit{Entropy}, 2013, 15.10: 4159-4187.
\bibitem{ivoLE2}BUKOVSKY, Ivo; KINSNER, Witold; HOMMA, Noriyasu. Learning Entropy as a Learning-Based Information Concept. \textit{Entropy}, 2019, 21.2: 166.
\bibitem{elbnd1}CEJNEK, Matous; BUKOVSKY, Ivo. Concept drift robust adaptive novelty detection for data streams. \textit{Neurocomputing}, 2018, 309: 46-53.
\bibitem{elbnd2} CEJNEK, Matous; BUKOVSKY, Ivo. Influence of type and level of noise on the performance of an adaptive novelty detector. In: \textit{2017 IEEE 16th International Conference on Cognitive Informatics \& Cognitive Computing (ICCI* CC)}. IEEE, 2017. p. 373-377.
\bibitem{elbnd3}CEJNEK, Matous; BUKOVSKY, Ivo; VYSATA, Oldrich. Adaptive classification of EEG for dementia diagnosis. In: \textit{2015 International Workshop on Computational Intelligence for Multimedia Understanding (IWCIM)}. IEEE, 2015. p. 1-5.
\bibitem{mackey}MACKEY, Michael C.; GLASS, Leon. Oscillation and chaos in physiological control systems. \textit{Science}, 1977, 197.4300: 287-289.
\bibitem{dead}SPANGENBERG, Mariana, et al. Detection of variance changes and mean value jumps in measurement noise for multipath mitigation in urban navigation. \textit{Navigation}, 2010, 57.1: 35-52.
\bibitem{stepchange}L'ECUYER, Pierre. History of uniform random number generation. In: \textit{2017 Winter Simulation Conference (WSC)}. IEEE, 2017. p. 202-230.
\bibitem{diagnosis}MAURYA, Mano Ram; RENGASWAMY, Raghunathan; VENKATASUBRAMANIAN, Venkat. Fault diagnosis using dynamic trend analysis: A review and recent developments. \textit{Engineering Applications of artificial intelligence}, 2007, 20.2: 133-146.
\bibitem{roc_orig}EGAN, James P. \textit{Signal detection theory and ROC-analysis}. Academic press, 1975.
\bibitem{roc_bible}FAWCETT, Tom. An introduction to ROC analysis. \textit{Pattern recognition letters}, 2006, 27.8: 861-874.
\bibitem{fault}GERTLER, Janos. \textit{Fault detection and diagnosis in engineering systems}. CRC press, 1998.
\bibitem{data_streams}YU, Kangqing, et al. Real-time Outlier Detection over Streaming Data. In: \textit{2019 IEEE SmartWorld, Ubiquitous Intelligence \& Computing, Advanced \& Trusted Computing, Scalable Computing \& Communications, Cloud \& Big Data Computing, Internet of People and Smart City Innovation (SmartWorld/SCALCOM/UIC/ATC/CBDCom/IOP/SCI)}. IEEE, 2019. p. 125-132.
\bibitem{surveilance}
RAMEZANI, Ramin; ANGELOV, Plamen; ZHOU, Xiaowei. A fast approach to novelty detection in video streams using recursive density estimation. In: \textit{2008 4th International IEEE Conference Intelligent Systems. IEEE, 2008}. p. 14-2-14-7.
\bibitem{robotics_marslan}
MARSLAND, Stephen; NEHMZOW, Ulrich; SHAPIRO, Jonathan. On-line novelty detection for autonomous mobile robots. \textit{Robotics and Autonomous Systems}, 2005, 51.2-3: 191-206.
\bibitem{robotics}
NEHMZOW, Ulrich, et al. Novelty detection as an intrinsic motivation for cumulative learning robots. In: \textit{Intrinsically Motivated Learning in Natural and Artificial Systems}. Springer, Berlin, Heidelberg, 2013. p. 185-207.
\bibitem{python}VAN ROSUM, G.; DRAKE, F. L. \textit{Python 3 Reference Manual}. Scotts Valley, CA: CreateSpace, 2009.
\bibitem{numpy}OLIPHANT, Travis E. \textit{A guide to NumPy (Vol. 1)}. Trelgol Publishing USA, 2006.
\bibitem{scipy}VIRTANEN, P. et al.  SciPy 1.0: Fundamental Algorithms for Scientific Computing in Python. \textit{Nature Methods}, 2020, 17(3), 261-272.
\end{thebibliography}
\cleardoublepage
\thispagestyle{empty}\

\newpage
\end{document}